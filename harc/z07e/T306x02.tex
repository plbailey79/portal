\documentclass{amsart}

%Includable Macros

%Packages
\usepackage{amssymb}
\usepackage{euscript}
\usepackage{doc}
\usepackage{verbatim}
\usepackage{amscd}

%Fonts
\newcommand{\scr}[1]{\EuScript{#1}}
\newcommand{\frk}[1]{\mathfrak{#1}}
\newcommand{\kal}[1]{\mathcal{#1}}
\newcommand{\bbb}[1]{\mathbb{#1}}
\newcommand{\tty}[1]{\texttt{#1}}
\newcommand{\NAME}[1]{\text{\rm{#1}}}
\newcommand{\BOLD}[1]{\text{\bf{#1}}}
\newcommand{\SANS}[1]{\text{\sf{#1}}}
\newcommand{\VERB}[1]{\text{\tt{#1}}}
\newcommand{\PERM}[1]{\VERB{\small{#1}}}

%Declarations
\theoremstyle{plain}
\newtheorem{Axm}{Axiom}
\newtheorem{Lem}{Lemma}
\newtheorem{Thm}{Theorem}
\newtheorem{Cor}{Corollary}
\newtheorem{Ass}{Assertion}
\newtheorem{Prop}{Proposition}
\newtheorem{Conj}{Conjecture}
\newtheorem{Res}{Result}

\theoremstyle{definition}
\newtheorem{Dec}{Declaration}
\newtheorem{Def}{Definition}
\newtheorem{Rem}{Remark}
\newtheorem{Obs}{Observation}
\newtheorem{Exm}{Example}
\newtheorem{Exc}{Exercise}
\newtheorem{Arg}{Argument}
\newtheorem{Type}{Type}
\newtheorem{Prob}{Problem}
\newtheorem{Prog}{Program}
\newtheorem{Algo}{Algorithm}
\newtheorem{Fact}{Fact}
\newtheorem{Warn}{Warning}
\newtheorem{Hint}{Hint}
\newtheorem{Ques}{Question}
\newtheorem{Sol}{Solution}
\newtheorem{Ans}{Answer}
\newtheorem{Asn}{Assignment}

\theoremstyle{remark}
\newtheorem{Not}{Notation}

%Quick Fonts

\newcommand{\cC}{\cal{C}}
\newcommand{\cH}{\cal{H}}
\newcommand{\cJ}{\cal{J}}
\newcommand{\cO}{\cal{O}}
\newcommand{\cU}{\cal{U}}

\newcommand{\bA}{\bbb{A}}
\newcommand{\bB}{\bbb{B}}
\newcommand{\bC}{\bbb{C}}
\newcommand{\bD}{\bbb{D}}
\newcommand{\bE}{\bbb{E}}
\newcommand{\bF}{\bbb{F}}
\newcommand{\bG}{\bbb{G}}
\newcommand{\bH}{\bbb{H}}
\newcommand{\bI}{\bbb{I}}
\newcommand{\bJ}{\bbb{J}}
\newcommand{\bK}{\bbb{K}}
\newcommand{\bL}{\bbb{L}}
\newcommand{\bM}{\bbb{M}}
\newcommand{\bN}{\bbb{N}}
\newcommand{\bO}{\bbb{O}}
\newcommand{\bP}{\bbb{P}}
\newcommand{\bQ}{\bbb{Q}}
\newcommand{\bR}{\bbb{R}}
\newcommand{\bS}{\bbb{S}}
\newcommand{\bT}{\bbb{T}}
\newcommand{\bU}{\bbb{U}}
\newcommand{\bV}{\bbb{V}}
\newcommand{\bW}{\bbb{W}}
\newcommand{\bX}{\bbb{X}}
\newcommand{\bY}{\bbb{Y}}
\newcommand{\bZ}{\bbb{Z}}
\newcommand{\bk}{\bold{k}}

% Moduli Spaces (Caligraphic}
\newcommand{\kA}{\kal{A}}
\newcommand{\kB}{\kal{B}}
\newcommand{\kC}{\kal{C}}
\newcommand{\kD}{\kal{D}}
\newcommand{\kE}{\kal{E}}
\newcommand{\kF}{\kal{F}}
\newcommand{\kG}{\kal{G}}
\newcommand{\kH}{\kal{H}}
\newcommand{\kI}{\kal{I}}
\newcommand{\kJ}{\kal{J}}
\newcommand{\kK}{\kal{K}}
\newcommand{\kL}{\kal{L}}
\newcommand{\kM}{\kal{M}}
\newcommand{\kN}{\kal{N}}
\newcommand{\kO}{\kal{O}}
\newcommand{\kP}{\kal{P}}
\newcommand{\kQ}{\kal{Q}}
\newcommand{\kR}{\kal{R}}
\newcommand{\kS}{\kal{S}}
\newcommand{\kT}{\kal{T}}
\newcommand{\kU}{\kal{U}}
\newcommand{\kV}{\kal{V}}
\newcommand{\kW}{\kal{W}}
\newcommand{\kX}{\kal{X}}
\newcommand{\kY}{\kal{Y}}
\newcommand{\kZ}{\kal{Z}}

% Collections of Sets (Script}
\newcommand{\sA}{\scr{A}}
\newcommand{\sB}{\scr{B}}
\newcommand{\sC}{\scr{C}}
\newcommand{\sD}{\scr{D}}
\newcommand{\sE}{\scr{E}}
\newcommand{\sF}{\scr{F}}
\newcommand{\sG}{\scr{G}}
\newcommand{\sH}{\scr{H}}
\newcommand{\sI}{\scr{I}}
\newcommand{\sJ}{\scr{J}}
\newcommand{\sK}{\scr{K}}
\newcommand{\sL}{\scr{L}}
\newcommand{\sM}{\scr{M}}
\newcommand{\sN}{\scr{N}}
\newcommand{\sO}{\scr{O}}
\newcommand{\sP}{\scr{P}}
\newcommand{\sQ}{\scr{Q}}
\newcommand{\sR}{\scr{R}}
\newcommand{\sS}{\scr{S}}
\newcommand{\sT}{\scr{T}}
\newcommand{\sU}{\scr{U}}
\newcommand{\sV}{\scr{V}}
\newcommand{\sW}{\scr{W}}
\newcommand{\sX}{\scr{X}}
\newcommand{\sY}{\scr{Y}}
\newcommand{\sZ}{\scr{Z}}

% Categories (fraktur}
\newcommand{\fA}{\frk{A}}
\newcommand{\fB}{\frk{B}}
\newcommand{\fC}{\frk{C}}
\newcommand{\fD}{\frk{D}}
\newcommand{\fE}{\frk{E}}
\newcommand{\fF}{\frk{F}}
\newcommand{\fG}{\frk{G}}
\newcommand{\fH}{\frk{H}}
\newcommand{\fI}{\frk{I}}
\newcommand{\fJ}{\frk{J}}
\newcommand{\fK}{\frk{K}}
\newcommand{\fL}{\frk{L}}
\newcommand{\fM}{\frk{M}}
\newcommand{\fN}{\frk{N}}
\newcommand{\fO}{\frk{O}}
\newcommand{\fP}{\frk{P}}
\newcommand{\fQ}{\frk{Q}}
\newcommand{\fR}{\frk{R}}
\newcommand{\fS}{\frk{S}}
\newcommand{\fT}{\frk{T}}
\newcommand{\fU}{\frk{U}}
\newcommand{\fV}{\frk{V}}
\newcommand{\fW}{\frk{W}}
\newcommand{\fX}{\frk{X}}
\newcommand{\fY}{\frk{Y}}
\newcommand{\fZ}{\frk{Z}}

%Itemize Commands
\newcommand{\Item}[1]{\item[\BOLD{#1.}]}
\newcommand{\ITEM}[1]{\item[\BOLD{(#1)}]}
\newcommand{\PROP}[1]{\NAME{\BOLD{(#1)}}}

%General
\renewcommand{\th}[1]{#1^{\NAME{th}}}
\newcommand{\sect}{\S}

%Logic
\newcommand{\NOT}  {\neg}
\newcommand{\AND}  {\wedge}
\newcommand{\OR}   {\vee}
\newcommand{\IMP}  {\Rightarrow}
\newcommand{\IFF}  {\Leftrightarrow}
\newcommand{\XOR}  {\updownarrow}
\newcommand{\NOR}  {\uparrow}
\newcommand{\NAND} {\downarrow}
\newcommand{\IMB}  {\Leftarrow}

%Set Theory
\newcommand{\EMPTY}{\varnothing}
\newcommand{\cross}{\times}
\newcommand{\remove}{\smallsetminus}
\newcommand{\symdif}{\triangle}
\newcommand{\restr}{\upharpoonright}
\newcommand{\embed}{\hookrightarrow}
\newcommand{\corresp}{\leftrightarrow}
\newcommand{\img}{\NAME{img}}
\newcommand{\dom}{\NAME{dom}}
\newcommand{\range}{\NAME{range}}
\newcommand{\rng}{\NAME{rng}}
\newcommand{\inc}{\NAME{inc}}
\newcommand{\id}{\NAME{id}}
\newcommand{\disj}{\sqcup}

%Group Theory
\newcommand{\normal}{\triangleleft}
\newcommand{\charac}{\lessdot}
\newcommand{\semi}{\rtimes}
\newcommand{\VEC}{\vec}
\newcommand{\BAR}{\overline}
\newcommand{\HAT}{\widehat}
\newcommand{\TIL}{\widetilde}
\newcommand{\CHK}{\check}
\newcommand{\gen}[1]{\langle #1 \rangle}
\newcommand{\ord}{\NAME{ord}}
\newcommand{\lcm}{\NAME{lcm}}
%\renewcommand{\gcd}{\NAME{gcd}}
%\renewcommand{\ker}{\NAME{ker}}

\newcommand{\iso}{\cong}
\newcommand{\homeo}{\approx}

\newcommand{\Sym}{\NAME{Sym}}
\newcommand{\Stb}{\NAME{Stb}}
\newcommand{\Orb}{\NAME{Orb}}
\newcommand{\Fix}{\NAME{Fix}}
\newcommand{\inn}{\NAME{inn}}
\newcommand{\ad}{\NAME{ad}}

\newcommand{\Cen}{\NAME{Cen}}

\newcommand{\Aut}{\NAME{Aut}}
\newcommand{\Inn}{\NAME{Inn}}
\newcommand{\Out}{\NAME{Out}}
\newcommand{\End}{\NAME{End}}
\newcommand{\Hom}{\NAME{Hom}}
\newcommand{\Iso}{\NAME{Iso}}
\newcommand{\Homeo}{\NAME{Homeo}}
\newcommand{\Diffeo}{\NAME{Diffeo}}

%Rings
\renewcommand{\char}{\NAME{char}}       % Characteristic
%\renewcommand{\deg}{\NAME{deg}}
\newcommand{\ideal}{\normal}
\newcommand{\divides}{\mid}
\newcommand{\prideal}{\overset{\NAME{prime}}{\ideal}}
\newcommand{\maxdeal}{\overset{\NAME{max}}{\ideal}}
\newcommand{\ENT}{\bullet}
\newcommand{\MOD}[1]{\NAME{ (mod }{#1}\NAME{)}}

%Modules
\newcommand{\ann}{\NAME{ann}}
\newcommand{\PIT}{\NAME{pit}}
\newcommand{\Fre}{\NAME{Fre}}
\newcommand{\Tor}{\NAME{Tor}}
\newcommand{\Ext}{\NAME{Ext}}

%Fields
\newcommand{\Gal}{\NAME{Gal}}
\newcommand{\Inv}{\NAME{Inv}}
\newcommand{\Chr}{\NAME{Chr}}
\newcommand{\trdeg}{\NAME{trdeg}}

%Linear Algebra
\DeclareMathOperator{\SPAN}{\NAME{span}}
\DeclareMathOperator{\col}{\NAME{col}}
\DeclareMathOperator{\row}{\NAME{row}}

%Matrices
\newcommand{\diag}{\NAME{diag}}
\newcommand{\rank}{\NAME{rank}}
\newcommand{\trc}{\NAME{trc}}

%Matrix Groups
\newcommand{\GL}{\BOLD{G}\BOLD{L}}
\newcommand{\SL}{\BOLD{S}\BOLD{L}}
\newcommand{\AL}{\BOLD{A}\BOLD{L}}
\newcommand{\DL}{\BOLD{D}\BOLD{L}}
\newcommand{\ZL}{\BOLD{Z}\BOLD{L}}
\newcommand{\PGL}{\BOLD{P}\BOLD{G}\BOLD{L}}
\newcommand{\PSL}{\BOLD{P}\BOLD{S}\BOLD{L}}
\newcommand{\GO}{\BOLD{G}\BOLD{O}}
\newcommand{\SO}{\BOLD{S}\BOLD{O}}
\newcommand{\PSO}{\BOLD{P}\BOLD{S}\BOLD{O}}
\newcommand{\GU}{\BOLD{G}\BOLD{U}}
\newcommand{\SU}{\BOLD{S}\BOLD{U}}
\newcommand{\SP}{\BOLD{S}\BOLD{P}}

\renewcommand{\det}{\NAME{det}}
\newcommand{\abs}[1]{|#1|}
\newcommand{\modu}[1]{|#1|}
\newcommand{\norm}[1]{\|#1\|}
\def\inner<#1,#2>{\langle #1, #2 \rangle}

%Analysis
\newcommand{\grad}{\triangledown}
\newcommand{\tensor}{\otimes}
\newcommand{\tensym}{\OPR}

%Vector Analysis
\newcommand{\proj}{\NAME{proj}}
\newcommand{\para}{\Vert}


\newcommand{\res}{\xi}

\begin{document}

\title{Abstract Algebra \\ Topic 2: Integers}
\author[P. L. Bailey]{Paul L. Bailey}
\address{Department of Mathematics and CSci\\
         Southern Arkansas University}
\email{plbailey@saumag.edu}
\date{\today}

\maketitle

\section{The Well-Ordering Principle}

The set of {\em natural numbers} is $\bN = \{ 0,1,2,3,\dots\}$,
as characterized by the five {\em Peano axioms}.  The main axiom
with which we are concerned is as follows.

\begin{Prop} \BOLD{(Peano's Axiom)} \\
Let $S \subset \bN$. If
\begin{itemize}
\ITEM{a} $0 \in S$, and \ITEM{b} $n \in S \IMP n+1 \in S$,
\end{itemize}
then $S = \bN$.
\end{Prop}

From this, we are able to develop two related tools for proving
many properties of the integers.  These tools are known as the Well-Ordering Principle,
which says that every nonempty set of natural numbers has a smallest element,
and the Induction Principle, which says that if we have a sequence of propositions
where the first is true and others follow from the previous one,
then they are all true.

\begin{Prop}
\BOLD{(Well-Ordering Principle)} \\
Let $X \subset \bN$ be nonempty. Then there exists $a \in X$ such
that $a \le x$ for every $x \in X$.
\end{Prop}

\begin{proof}
Let $X \subset \bN$ and assume that $X$ has no smallest element;
we show that $X = \EMPTY$. Let
\[ S = \{ n \in \bN \mid n < x \text{ for every } x \in X \} . \]
Clearly $S \cap X = \EMPTY$; if we show that $S = \bN$, then $X =
\EMPTY$.

Since $0$ is less than or equal to every natural number, $0$ is
less than or equal to every natural number in $X$.  Since $X$ has
no smallest element, $0 \notin X$, so $0 < x$ for every $x \in X$.
Thus $0 \in S$.

Suppose that $n \in S$. Then $n < x$ for every $x \in X$, so $n+1
\le x$ for every $x \in X$. If $n+1$ were in $X$, it would be the
smallest element of $X$; since $X$ has no smallest element, $n+1
\notin X$; thus $n+1 \ne x$ for every $x \in X$, whence $n+1 < x$
for every $x \in X$. It follows that $n+1 \in S$, and by Peano's
Axiom, $S = \bN$.
\end{proof}

\newpage

\section{The Induction Principles}

\begin{Prop} \BOLD{(Induction Principle)} \\
Let $\{ p_i \mid i \in \bN \}$ be a set of propositions indexed by
$\bN$. Suppose that
\begin{itemize}
\ITEM{I1} $p_0$ is true; \ITEM{I2} $p_{n-1}$ implies $p_n$, for
$n>0$.
\end{itemize}
Then $p_i$ is true for all $i \in \bN$.
\end{Prop}

\begin{proof}
Suppose not, and let $n \in \bN$ be the smallest natural number
such that $p_n$ is false.  Then $n \ne 0$, since $p_0$ is true by
\PROP{I1}, so $n-1$ exists as a natural number.  Since $n-1 < n$,
$p_{n-1}$ is true. By \PROP{I2}, $p_{n-1} \IMP p_{n}$, so $p_{n}$
is true, contradicting the assumption. Thus $p_i$ is true for all
$i \in \bN$.
\end{proof}

We call \PROP{I1} the {\em base case} and \PROP{I2} the {\em
inductive step}. We note that by shifting, we can actually start
the induction at any integer. Here is an example demonstrating
proof by induction.

\begin{Exm}
Show that $11^n - 4^n$ is a multiple of $7$ for all $n \in \bN$.
\end{Exm}

\begin{proof}
A natural number $a$ is a multiple of $7$ if and only if $a = 7b$
for some natural number $b$. We proceed by induction on $n$.
First we verify the base case, when $n = 0$, and then demonstrate
the induction step, wherein we show that if the proposition is
true for $n-1$, then it is true for $n$.

\PROP{I1} Let $n=0$.  Then $n = 7 \cdot 0$, so $n$ is a multiple
of $7$ in this case. This verifies the base case.

\PROP{I2} Let $n > 0$, and assume that $11^{n-1} - 4^{n-1}$ is a
multiple of $7$. Then $11^{n-1} - 4^{n-1} = 7k$ for some $k \in
\bN$. Now compute
\begin{align*}
11^n - 4^n
    &= 11^n - 11 \cdot 4^{n-1} + 11 \cdot 4^{n-1} - 4^{n} \\
    &= 11(11^{n-1} - 4^{n-1}) + 4^{n-1}(11-4) \\
    &= 11 \cdot 7k + 4^{n-1} \cdot 7 \\
    &= 7(11k+4^{n-1}) ,
\end{align*}
which is a multiple of seven.

Thus properties \PROP{I1} and \PROP{I2} hold, so the proposition
is true for all $n \in \bN$.
\end{proof}

\begin{Prop} \BOLD{(Strong Induction Principle)} \\
Let $\{ p_i \mid i \in \bN \}$ be a set of propositions indexed by
$\bN$. Suppose that
\begin{itemize}
\ITEM{IS} if $p_i$ is true for all $i < n$, then $p_{n}$ is true.
\end{itemize}
Then $p_i$ is true for all $i \in \bN$.
\end{Prop}

\begin{proof}
Suppose not, and let $m$ be the smallest natural number such that $p_m$ is false.
Then $p_i$ is true for all $i < m$. By
\PROP{IS}, $p_m$ is true, contradicting the assumption. Thus $p_i$
is true for all $i \in \bN$.
\end{proof}

It is common in the statement of the strong induction principle to
include the base case \PROP{I1}, that $p_0$ is true, as a premise.
In practice, we may have to verify \PROP{I1} as a step in demonstrating \PROP{IS}.
We note that \PROP{I1} is implied by \PROP{IS}, but that \PROP{I2}
is not implied by \PROP{IS} (why?).

\newpage

\section{The Division Algorithm}

\begin{Prop}
\BOLD{(Division Algorithm)} \\
Let $m,n \in \bZ$ with $m \ne 0$. There exist unique integers $q,r
\in \bZ$ such that
\[ n = qm + r \qquad \text{ and } \qquad 0 \le r < \modu{m} . \]
\end{Prop}

We offer two proofs of this, one using the well-ordering principle
directly, and the other phrased in terms of strong induction.

\begin{proof}[Proof by Well-Ordering]
First assume that $m$ and $n$ are positive.

Let $X = \{ z \in \bZ \mid z = n - km \text{ for some } k \in \bZ
\}$. The subset of $X$ consisting of nonnegative integers is a
subset of $\bN$, and by the Well-Ordering Principle, contains a
smallest member, say $r$.  That is, $r = n - qm$ for some $q \in
\bZ$, so $n = qm + r$. We know $0 \le r$.  Also, $r < m$, for
otherwise, $r - m$ is positive, less than $r$, and in $X$.

For uniqueness, assume $n = q_{1}m + r_{1}$ and $n = q_{2} m +
r_{2}$, where $q_{1},r_{1},q_{2},r_{2} \in \bZ$, $0 \le r_{1} <
m$, and $0 \le r_{2} < m$. Then $m(q_{1}-q_{2}) = r_{1}-r_{2}$;
also $-m < r_{1}-r_{2} < m$. Since $m \divides (r_{1}-r_{2})$, we
must have $r_{1} - r_{2} = 0$. Thus $r_{1} = r_{2}$, which forces
$q_{1} = q_{2}$.

The proposition remains true if one or both of the original
numbers are negative because, if $n = mq+r$ with $0 \le r < m$,
then $0 \le m-r < m$ when $r>0$, and
\begin{itemize}
\item $(-n) = m(-q-1) + (m-r)$ if $r>0$ and $(-n) = m(-q)$ if $r =
0$; \item $(-n) = (-m)(q+1) + (m-r)$ if $r > 0$ and $(-n) = (-m)q$
if $r = 0$; \item $n = (-m)(-q) + r$.
\end{itemize}
\end{proof}

\begin{proof}[Proof by Strong Induction]
Assume that $m$ and $n$ are positive.

If $m > n$, set $q=0$ and $r=n$. If $m = n$, set $q = 1$ and $r = 0$.
Otherwise, we have $0 < m < n$.
Proceed by strong induction on $n$. Here we assume that the
proposition is true for all natural number less that $n$, and show
that this implies that the proposition is true for $n$.  Then, by
the conclusion of the Strong Induction Principle, the proposition
will be true for all natural numbers $n$.

Note that $n = m + (n-m)$ and $n-m < n$, so by induction, $n-m =
mq_1 + r$ for some $q_1,r \in \bZ$ with $0 \le r_1 < m$.
Therefore $n = m(q_1 + 1) + r_1$; set $q = q_1 + 1$ to see that $n
= mq + r$, with $r$ still in the range $0 \le r < m$.

The proof for uniqueness and the cases where $m$ and/or $n$ are
negative are the same as above.
\end{proof}

Notice that the proof by induction reveals division as repeated subtraction.
It more closely mimics the algorithm we use to find $q$ and $r$ than
does the proof via the Well-Ordering Principle.

\newpage

\section{The Euclidean Algorithm}

\begin{Def}
Let $m,n \in \bZ$.  We say that $m$ {\em divides} $n$, and write
$m \divides n$, if there exists an integer $k$ such that $n = km$.
\end{Def}

\begin{Def}
Let $m,n \in \bZ$ be nonzero.  We say that a positive integer $d \in \bZ$
is a {\em greatest common divisor} of $m$ and $n$,
and write $d = \gcd(m,n)$, if
\begin{itemize}
\ITEM{a} $d \divides m$ and $d \divides n$;
\ITEM{b} $e \divides m$ and $e \divides n$ implies $e \divides d$, for all $e \in \bZ$.
\end{itemize}
\end{Def}

\begin{Prop} \BOLD{(Euclidean Algorithm)} \\
Let $m,n \in \bZ$ be nonzero. Then there exists a unique $d \in
\bZ$ such that $d = \gcd(m,n)$, and there exist integers $x,y \in
\bZ$ such that
\[ d = xm + yn . \]
\end{Prop}

\begin{proof}
Let $X = \{ z \in \bZ \mid z = xm + yn \text{ for some } x,y \in
\bZ \}$. Then the subset of $X$ consisting of positive integers
contains a smallest member, say $d$, where $d = xm + yn$ for some
$x,y \in \bZ$.

Now $m = qd + r$ for some $q,r \in \bZ$ with $0 \le r < d$. Then
$m = q(xm+yn) + r$, so $r = (1 - qxm)m + (qy)n \in X$. Since $r <
d$ and $d$ is the smallest positive integer in $X$, we have $r =
0$.  Thus $d \divides m$.  Similarly, $d \divides n$.

If $e \divides m$ and $e \divides n$, then $m = ke$ and $n = le$
for some $k,l \in \bZ$. Then $d = xke + yle = (xk+yl)e$.
Therefore $e \divides d$. This shows that $d = \gcd(m,n)$.

For uniqueness of a greatest common divisor, suppose that $e$ also
satisfies the conditions of a $\gcd$.  Then $d \divides e$ and $e
\divides d$. Thus $d = ie$ and $e = jd$ for some $i,j \in \bZ$.
Then $d = ijd$, so $ij = 1$.  Since $i$ and $j$ are integers, then
$i = \pm 1$.  Since $d$ and $e$ are both positive, we must have $i
= 1$.  Thus $d = e$.
\end{proof}

This shows that the $d = \gcd(m,n)$ exists and the formula $xm+yn = d$ holds,
but does not give a method of finding $x$, $y$, and $d$.  The method we develop is based
on the following propositions.

\begin{Prop} \label{GCDObvious}
Let $m,n \in \bN$ and suppose that $m \divides n$.
Then $\gcd(m,n) = m$.
\end{Prop}

\begin{proof}
Clearly $m \divides m$, and we are given $m \divides n$.
Now suppose that $e \divides m$ and $e \divides n$.
Then $e \divides m$.  Thus $m = \gcd(m,n)$.
\end{proof}

\begin{Prop} \label{GCDRecursive}
Let $m,n \in \bZ$ be nonzero, and let $q,r \in \bZ$ such that $n =
qm + r$. Then $\gcd(n,m) = \gcd(m,r)$.
\end{Prop}

\begin{proof}
Let $d = \gcd(n,m)$.  We wish to show that $d = \gcd(m,r)$, which
requires showing that $d$ satisfies the two properties of being the
greatest common divisor of $m$ and $r$.

Since $d = \gcd(n,m)$, we know that $d \divides n$ and $d \divides
m$. Thus $n = ad$ and $m = bd$ for some $a,b \in \bZ$. Now $r = n
- mq = ad - bdq = d(a - bq)$, so $d \divides r$.  Thus $d$ is a
common divisor of $m$ and $r$.

Let $e \in \bZ$ such that $e \divides m$ and $e \divides r$. Then
$m = ge$ and $n = he$ for some $g,h \in \bZ$, so $n = geq + he =
e(gq + h)$; thus $e \divides n$, so $e$ is a common divisor of $n$
and $m$. Since $d = \gcd(n,m)$, $e \divides d$.  Therefore, $d =
\gcd(m,r)$.
\end{proof}

\newpage

There is an efficient effective procedure for finding the greatest
common divisor of two integers.  It is based on the following
proposition.

Now let $m,n \in \bZ$ be arbitrary integers, and write $n = mq +
r$, where $0 \le r < m$. Let $r_{0} = n$, $r_{1} = m$, $r_{2} =
r$, and $q_{1} = q$. Then the equation becomes $r_{0} = r_{1}q_{1}
+ r_{2}$. Repeat the process by writing $m = rq_{2} + r_{3}$,
which is the same as $r_{1} = r_{2}q_{2} + r_{3}$, with $0 \le
r_{3} < r_{2}$. Continue in this manner, so in the $\th{i}$ stage,
we have $r_{i-1} = r_{i}q_{i} + r_{i+1}$, with $0 \le r_{i+1} <
r_{i}$. Since $r_{i}$ keeps getting smaller, it must eventually
reach zero.

Let $k$ be the smallest integer such that $r_{k+1} = 0$. By the
above proposition and induction,
\[ \gcd(n,m) = \gcd(m,r) = \dots = \gcd(r_{k-1},r_{k}) . \]
But $r_{k-1} = r_{k}q_{k} + r_{k+1} = r_{k}q_{k}$. Thus $r_{k}
\divides r_{k-1}$, so $\gcd(r_{k-1},r_{k}) = r_{k}$. Therefore
$\gcd(n,m) = r_{k}$.  This process for finding the $\gcd$ is known
as the {\em Euclidean Algorithm}.

In order to find the unique integers $x$ and $y$ such that $xm +
yn = \gcd(m,n)$, use the equations derived above and work
backward. Start with $r_{k} = r_{k-2} - r_{k-1}q_{k-1}$.
Substitute the previous equation $r_{k-1} = r_{k-3} -
r_{k-2}q_{k-2}$ into this one to obtain
\[ r_{k} = r_{k-2} - (r_{k-3} - r_{k-2}q_{k-2})q_{k-1}
        = r_{k-2}(q_{k-2}q_{k-1}+1) - r_{k-3}q_{k-1} . \]
Continuing in this way until you arrive back at the beginning.

\begin{Exm}
Let $n = 210$ and $m = 165$. Work forward to find the $\gcd$:
\begin{itemize}
\item $210 = 165 \cdot 1 + 45$; \item $165 = 45 \cdot 3 + 30$;
\item $45 = 30 \cdot 1 + 15$; \item $30 = 15 \cdot 2 + 0$.
\end{itemize}
Therefore, $\gcd(210,165) = 15$.  Now work backwards to find the
coefficients:
\begin{itemize}
\item $15 = 45 - 30 \cdot 1$; \item $15 = 45 - (165 - 45 \cdot 3)
= 45 \cdot 4 - 165$; \item $15 = (210 - 165) \cdot 4 - 165 = 210
\cdot 4 - 165 \cdot 5$.
\end{itemize}
Therefore, $15 = 210 \cdot 4 + 165 \cdot (-5)$.
\end{Exm}

Let's briefly analyze the inductive process of ``working
backwards''.

At each stage, let $m$ denote the smaller number and let $n$ denote the larger number.
Always attach $x$ to $m$ and $y$ to $n$, to get $d = xm + yn$, where $d = \gcd(m,n)$.
Now at the very end, the remainder is zero, so
$n = mq + 0$.
Thus $m = \gcd(n,m)$, that is, $d = m$.
Writing $d$ as a linear combination at this stage, we have
\[ d = (1)m + (0) n m \]
so $x = 1$ and $y = 0$.

Now we want to lift this to a previous equation of the form $n = mq + r$.
Assume, by way of induction, that we have already lifted it to the
next equation; that is, we have $n' = m' q' + r'$,
where $n' = m$, $m' = r$, and we can express $d$ as a linear combination of $m'$ and $n'$,
like this:
\[ d = x' m' + y' n' . \]
Then $d = x' r + y' m$.  Substitute in $r = n - mq$ to express $d$ as a linear combination
of $m$ and $n$; you get $d = x'(n - mq) + y' m = (y' - x'q)m + x' n$.
Set $x = y' - x'q$ and $y = x'$ to obtain $d = xm + yn$.

\newpage

%\section{Relative Primeness}
%
%An alternate method of find the greatest common divisor of two integers is to factor them
%into primes, and collect all of the common primes; their product will be the greatest
%common divisor.  If the integers have no primes in common, then their greatest common
%divisor is one.

\begin{Def}
Let $m,n \in \bZ$.  We say that $m$ and $n$ are {\em relatively prime} if
\[ \gcd(m,n) = 1 . \]
\end{Def}

\begin{Prop} \label{GCDOne}
Let $m,n \in \bZ$.
Then
\[ \gcd(m,n) = 1 \quad \IFF \quad xm + yn = 1 \text{ for some $x,y \in \bZ$} . \]
\end{Prop}

\begin{proof}
We have already seen that if $\gcd(m,n) = 1$, then $xm + yn = 1$ for some $x,y \in \bZ$.
Thus we prove the reverse direction; suppose that $xm + yn = 1$ for some $x,y \in \bZ$.
We wish to show that $\gcd(m,n) = 1$.

Clearly $1 \divides m$ and $1 \divides n$.
Suppose that $e \divides m$ and $e \divides n$.
Then $m = ke$ and $n = le$ for some $k,l \in e$.
So
\[ 1 = xke + yle = (xk+yl)e. \]
Thus $e \divides 1$, whence $\gcd(m,n) = 1$.
\end{proof}

%One must be careful, when proving theoretical results about numbers, that one does not
%inappropriately move through the rational numbers (the arithmetic properties of the $\bQ$
%much different than those of $\bZ$).  So, we keep track of whether
%a given number is integral or rational.  Here,
%We note that
%if $d \divides a$, then $a = cd$ for some $c$; we let $\frac{a}{d}$ denote the integer
%which is $c$.

\begin{Prop}
Let $m,n,d \in \bZ$ such that $\gcd(m,n) = d$.  Then $\gcd(\frac{m}{d},\frac{n}{d}) = 1$.
\end{Prop}

\begin{proof}
Since $xm + yn = d$ for some $x,y \in \bZ$, we have $x\frac{m}{d} + y\frac{n}{d} = 1$.
From Proposition \ref{GCDOne},
we conclude that $\gcd(\frac{m}{d},\frac{n}{d}) = 1$.
\end{proof}

\begin{Prop}
Let $a,b,c \in \bZ$.  If $a \divides bc$ and $\gcd(a,b) = 1$, then $a \divides c$.
\end{Prop}

\begin{proof}
Since $a \divides bc$, there exists $z \in \bZ$ such that $az = bc$.
Since $\gcd(a,b) = 1$, there exist $x,y \in \bZ$ such that $xa + yb = 1$.
Multiplying both sides by $c$ gives
\[ xac + ybc = c \IMP xac + yaz = c \IMP a(xc+yz) = c . \]
Thus $a \divides c$.
\end{proof}

\begin{Prop}
Let $a,b,c \in \bZ$.  If $a \divides c$, $b \divides c$, and $\gcd(a,b) = 1$, then $ab \divides c$.
\end{Prop}

\begin{proof}
There exist $e,f,x,y \in \bZ$ such that $ae = c$, $bf = c$, and $xa + yb = 1$.
Multiplying the last equation by $c$ gives $xac + ybc = c$.
Substitution gives $xabf + ybae = c$, so $ab(xf+ye) = c$.
Thus $ab \divides c$.
\end{proof}

\begin{Def}
Let $m,n \in \bZ$.  We say that a positive integer $l \in \bZ$ is a {\em least common multiple}
of $m$ and $n$, and write $l = \lcm(m,n)$, if
\begin{itemize}
\ITEM{a} $m \divides l$ and $n \divides l$;
\ITEM{b} $m \divides k$ and $n \divides k$ implies $l \divides k$, for all $k \in \bZ$.
\end{itemize}
\end{Def}

\begin{Prop}
Let $m,n \in \bZ$ be nonzero. Then there exists a unique $l \in \bZ$
such that $l = \lcm(m,n)$, and if $d = \gcd(m,n)$, then
\[ l = \frac{mn}{d} . \]
\end{Prop}

\begin{proof}
Let $l = \frac{mn}{d}$; we wish to show that $l$ is a least common multiple for $m$ and $n$.
Since $d = \gcd(m,n)$, $\frac{m}{d}$ and $\frac{n}{d}$ are integers,
and $l = m \frac{n}{d} = n \frac{m}{d}$. Thus $m \divides l$ and $n \divides l$.

Now suppose that $k$ is an integer such that $m \divides k$ and $n \divides k$;
we wish to show that $l \divides k$.
We have $k = ae$ and $k = bf$ for some $e,f \in \bZ$.  Thus $ae = bf$,
and dividing by $d$ gives $e \frac{a}{d} = f \frac{b}{d}$.
Thus $\frac{a}{d} \divides f \frac{b}{d}$, and since $\gcd(\frac{a}{d},\frac{b}{d}) = 1$,
we have $\frac{a}{d} \divides f$.
Thus $f = g \frac{a}{d}$ for some $g \in \bZ$, so $k = bf = g \frac{ab}{d} = gl$.
Thus $l \divides k$, so $l$ is a least common multiple of $m$ and $n$.

For uniqueness, note that any two least common multiples must divide each other;
but they are both positive, so they must be equal.
\end{proof}

\newpage

\section{Fundamental Theorem of Arithmetic}

\begin{Def}
An integer $p \ge 2$, is called {\em prime} if
\[ a \divides p \IMP a = 1 \text{ or } a = p, \quad \text{ where } a \in \bN . \]
\end{Def}

\begin{Prop}
Let $a,p \in \bZ$, with $p$ prime.  Then
\[ \gcd(a,p) =
\begin{cases}
    $p$ & \quad \text{ if $p \divides a$} ; \\
    $1$ & \quad \text{ otherwise}.
\end{cases} \]
\end{Prop}

\begin{proof}
Let $d = \gcd(a,p)$.  Then $d \divides p$, so $d = 1$ or $d = p$.
We have $p \divides p$, so if $p \divides a$, we have $p \divides d$.
In this case, $d = p$.
If $p$ does not divide $a$, then $d \ne p$, so we must have $d = 1$.
\end{proof}

\begin{Prop} \BOLD{(Euclid's Argument)} \\
Let $p \in \bZ$, $p \ge 2$. Then $p$ is prime if and only if
\[ p \divides ab \IMP p \divides a \text{ or } p \divides b, \quad \text{ where } a,b \in \bN . \]
\end{Prop}

\begin{proof}
\text{ } \\
($\IMP$) Given that $a \divides p \IMP a = 1 \text{ or } a = p$,
suppose that $p \divides ab$. Then there exists $k \in \bN$ such
that $kp = ab$. Suppose that $p$ does not divide $a$; then
$\gcd(a,p) = 1$. Thus there exist $x,y \in \bZ$ such that $xa + yp
= 1$. Multiply by $b$ to get $xab + ypb = b$. Substitute $kp$ for
$ab$ to get $(xk + yb)p = b$.
Thus $p \divides b$. \\
($\IMB$) Given that $p \divides ab \IMP p \divides a \text{ or } p
\divides b$, suppose that $a \divides p$. Then there exists $k \in
\bN$ such that $ak = p$. So $p \divides ak$, so $p \divides a$ or
$p \divides k$. If $p \divides a$, then $pl = a$ for some $l \in
\bN$, in which case $alk = a$ and $lk = 1$, which implies that $k
= 1$ so $a = p$. If $p \divides k$, then $k = pm$ for some $m \in
\bN$, and $apm = p$, so $am = 1$ which implies that $a = 1$.
\end{proof}

\begin{Prop}
Let $n \in \bZ$ with $n \ge 2$. \\
There exists a prime $p \in \bZ$ such that $p \divides n$.
\end{Prop}

\begin{proof}
Proceed by strong induction on $n$. If $n$ is prime, it divides
itself; otherwise, $n$ is not prime, and $n = ab$ for some $a,b
\in \bZ$ with $a < n$ and $b < n$. By induction, $a$ is divisible
by a prime, so $n = ab$ is divisible by that prime.
\end{proof}

\begin{Prop}\BOLD{(Fundamental Theorem of Arithmetic)} \\
Let $n \in \bZ$, $n \ge 2$.  Then there exist unique prime numbers
$p_1 , \dots , p_r$, unique up to order, such that
\[ n = \prod_{i=1}^r p_i . \]
\end{Prop}

\begin{proof}
We know that $n$ is divisible by some prime, say $n = pm$ for some
$p,m \in \bZ$ with $p$ prime. Since $m$ is smaller than $n$, we
conclude by induction that $m$ factors into a product of primes;
thus $n = pm$ factors into a product of primes.
To see that this factorization is unique,
suppose that there exist prime $p_1, \dots, p_r$ and $q_1, \dots,
q_s$ such that
\[ n = p_1 p_2 \cdots p_r = q_1 q_2 \cdots q_s . \]
By repeatedly applying Euclid's Argument, we see that $p_1
\divides q_i$ for some $i$, and by renumbering if necessary, we
may assume that $p_1 \divides q_1$. Since $q_1$ is prime, $p_1 =
1$ or $p_1 = q_1$; but $p_1$ is also prime, so it is greater than
$1$; thus $p_1 = q_1$.  Canceling these, we see that $p_2 \cdots
p_r = q_2 \cdots q_s$, and we may repeat this process obtaining
$p_2 = q_2$, $p_3 = q_3$, and so forth. We also see that $r = s$,
for otherwise, we would obtain an equation in which a product of
primes equals one.
\end{proof}

\begin{comment}
\begin{Prop}
Let $P = \{ n \in \bZ \mid n \text{ is prime} \}$. Then $P$ is
infinite.
\end{Prop}

\begin{proof}
Suppose that $P$ is finite; then $P = \{ p_1, \dots, p_r \}$ for
some primes $p_i$. Set
\[ n = 1 + \prod_{i=1}^r p_i . \]
Clearly $n > 1$, so $n$ is divisible by some prime $p$, and $p =
p_i$ for some $i$. Thus $p$ divides $n$ and $\prod_{i=1}^r p_i$,
so $p$ divides $1 = n - \prod_{i=1}^r p_i$. But $1$ cannot be
divisible by a prime, so we have a contradiction.
\end{proof}
\end{comment}

\newpage

\section{Congruence Modulo $n$}

\begin{Def}
Let $n \in \bN$, and define a relation $\equiv_{n}$ on $\bZ$ by
\[ a \equiv_n b \quad \IFF \quad n \divides (a-b) . \]
This relation is called {\em congruence modulo $n$};
that is, if $a \equiv_n b$, we say that $a$ is {\em congruent}
to $b$ modulo $n$.
This relation may also be written $a \equiv b \MOD{n}$, or simply $a \equiv b$
if the $n$ is understood.
\end{Def}

\begin{Prop}
Let $n \in \bN$. Then $\equiv$ modulo $n$ is an equivalence relation on $\bZ$.
\end{Prop}

\begin{proof}
We wish to show that $\equiv$ is reflexive, symmetric, and transitive.

({\em Reflexivity}) Let $a \in \bZ$. Now $0 \cdot n = 0 = a-a$; thus
$n \divides (a-a)$, so $a \equiv a$. Therefore $\equiv$ is
reflexive.

({\em Symmetry}) Let $a,b \in \bZ$. Suppose that $a \equiv b$; then
$n \divides (a-b)$. Then there exists $k \in \bZ$ such that $nk =
a-b$. Then $n(-k) = b-a$, so $n \divides (b-a)$. Thus $b \equiv a$.
Similarly, $b \equiv a \IMP a \equiv b$. Therefore $\equiv$ is
symmetric.

({\em Transitivity}) Let $a,b,c \in \bZ$, and suppose that $a \equiv
b$ and $b \equiv c$. Then $nk = a-b$ and $nl = b-c$ for some $k,l
\in \bZ$. Then $a-c = nk - nl = n(k-l)$, so $n \divides (a-c)$. Thus
$a \equiv c$. Therefore $\equiv$ is transitive.
\end{proof}

\begin{Prop} \label{UniquePreferred}
Let $n \in \bN$ and let $a_{1},a_{2} \in \bZ$. By the Division
Algorithm, there exist unique integers $q_{1},r_{1},q_{2},r_{2} \in
\bZ$ such that
\begin{itemize}
\item $a_{1} = nq_{1} + r_{1}$, where $0 \le r_{1} < n$;
\item $a_{2} = nq_{2} + r_{2}$, where $0 \le r_{2} < n$.
\end{itemize}
Then $a_{1} \equiv a_{2} \MOD{n}$ if and only if $r_{1} = r_{2}$.
\end{Prop}

\begin{proof}
\text{ }

($\IMP$) Suppose that $a_{1} \equiv a_{2}$. Then $n \divides (a_{1}
- a_{2})$. This means that $nk = a_{1} - a_{2}$ for some $k \in
\bZ$. But $a_{1} - a_{2} = n(q_{1}-q_{2}) + (r_{1} - r_{2})$. Then
$n(k + q_{1} - q_{2}) = r_{1} - r_{2}$, so $n \divides r_{1} -
r_{2}$.

Multiplying the inequality $0 \le r_{2} < n$ by $-1$
gives $-n < -r_{2} \le 0$.
Adding this inequality to the inequality $0 \le r_{1} < n$
gives $-n < r_{1}-r_{2} < n$.
But $r_{1} - r_{2}$ is an integer multiple of $n$;
the only possibility, then, is that $r_{1} - r_{2} = 0$.
Thus $r_{1} = r_{2}$.

($\IMB$)
Suppose that $r_{1} = r_{2}$.
Then $a_{1} - a_{2} = nq_{1} - nq_{2} = n(q_{1}-q_{2})$.
Thus $n \divides (a_{1} - a_{2})$,
so $a_{1} \equiv a_{2}$.
\end{proof}

\newpage

\section{Chinese Remainder Theorem}

The Chinese Remainder Theorem indicates a condition under which we can solve a system
of congruences.

\begin{Prop} \BOLD{(Chinese Remainder Theorem)} \\
Let $a,b,m,n \in \bZ$ such that $\gcd(m,n) = 1$.
Then there exists $c \in \bZ$ with $0 \le c < mn$ such that
\begin{itemize}
\item $c \equiv a \MOD{m}$;
\item $c \equiv b \MOD{n}$.
\end{itemize}
\end{Prop}

\begin{proof}
There exist $x,y \in \bZ$ such that $mx + ny = 1$.
Let $c = mxb + nya$.
Then
\[ c - a = mxb + nya - a = mxb + (ny-1)a = mxb - mxa, \]
so $m$ divides $c-a$; thus $c \equiv a \MOD{m}$.
Also
\[ c - b = mxb + nya - b = (mx-1)b + nya = -nyb + nya, \]
so $n$ divides $c-b$; thus $c \equiv b \MOD{n}$.
\end{proof}

\begin{Exm}
Let $m = 104$, $n = 231$, $a = 11$, and $b = 23$.
Find $c \in \bZ$ with $0 \le c < mn$ such that $c \equiv a \MOD{m}$ and $c \equiv b \MOD{n}$.
\end{Exm}

\begin{proof}[Solution]
First we use the Euclidean algorithm to write $mx + yn = d$.
We have
\begin{align*}
231 &= 104 \cdot 2 + 23 \\
104 &= 23 \cdot 4 + 12 \\
23 &= 12 \cdot 1 + 11 \\
12 &= 11 \cdot 1 + 1 \\
11 &= 1 \cdot 11 + 0
\end{align*}
Thus
\begin{align*}
1 &= (-1) 11 + 12 \\
  &= (2) 12 + (-1) 23 \\
  &= (-9) 23 + (2) 104 \\
  &= (20) 104 + (-9) 231
\end{align*}
That is, $x = 20$, $y = -9$, and $d = 1$,

Now set
\[ c = mxb + nya \MOD{24024} = 24971 \MOD{24024} = 947. \]
\end{proof}

\newpage

\section{Integers Modulo $n$}

Let $n \in \bZ$, $n \ge 2$.
The equivalence relation $\equiv_{n}$ partitions the set $\bZ$
into blocks, known as {\em congruence classes modulo $n$}.
For an integer $a \in \bZ$, denote its
congruence class by $[a]_{n}$.
If the $n$ is understood, we may write this congruence class as $[a]$,
or more commonly, as $\BAR{a}$.

An element $r \in \bZ$ is called a {\em preferred representative}
for $[a]_{n}$ if $r \in [a]_{n}$ and $0 \le r < n$.  This is the remainder
when any element in $[a]_n$ is divided by $n$

The division algorithm for the integers tells us that there is a
preferred representative for each congruence class. Also, Proposition \ref{UniquePreferred}
guarantees that as
$r$ ranges over the integers from $0$ to $n-1$, the congruence
classes $[r]_{n}$ are distinct.  Thus there are exactly $n$
equivalence classes, modulo $n$.  Henceforth, whenever we refer to $\bZ_n$,
assume that $n \in \bZ$ with $n \ge 2$.

\begin{Def}
The {\em ring of integers modulo $n$}
is
\[ \bZ_n = \{ [a]_n \mid a \in \bZ \}. \]
\end{Def}

That is, $\bZ_n$ is the set of equivalence classes modulo $n$, and $\modu{\bZ_n} = n$.
For example,
\[ \bZ_{7} = \{ \BAR{0}, \BAR{1}, \BAR{2}, \BAR{3},
        \BAR{4}, \BAR{5}, \BAR{6} \} . \]

\begin{Prop}
Define the binary operations on $\bZ_n$,
\[ + : \bZ_n \cross \bZ_n \to \bZ_n \quad \text{ and } \quad \cdot : \bZ_n \cross \bZ_n \to \bZ_n , \]
known as addition and multiplication, by
\[ \BAR{a} + \BAR{b} = \BAR{a+b}
        \quad \text{ and } \quad \BAR{a}\cdot\BAR{b} = \BAR{ab} . \]
These operations are well-defined.
\end{Prop}

\begin{proof}
Select $a_{1},a_{2},b_{1},b_{2} \in \bZ$ such that $a_{1} \equiv
a_{2}$ and $b_{1} \equiv b_{2}$; say $a_{1} - a_{2} = kn$ and $b_{1}
- b_{2} = ln$ for some $k,l \in \bZ$.

({\em Addition})
We wish to show that $\BAR{a_{1}+b_{1}} = \BAR{a_{2} + b_{2}}$, i.e.,
that $a_{1} + b_{1} \equiv a_{2} + b_{2}$.
We simply add the equations above to obtain
$a_{1} - a_{2} + b_{1} - b_{2} = kn + ln$;
thus
\[ (a_{1} + b_{1}) - (a_{2} + b_{2}) = (k+l)n ; \]
from this, $n \divides ((a_{1}+b_{1})-(a_{2}+b_{2}))$,
so $a_{1}+b_{1} \equiv a_{2} + b_{2}$.

({\em Multiplication})
We wish to show
that $\BAR{a_{1}}\cdot\BAR{b_{1}} = \BAR{a_{2}} \cdot \BAR{b_{2}}$, i.e.,
that $a_{1}b_{1} \equiv a_{2}b_{2}$.
To do this, adjust the original equations to obtain
%\[ a_{1} = a_{2} + kn \qquad \text{ and } \qquad b_{1} = b_{2} + ln \]
$a_{1} = a_{2} + kn$ and $b_{1} = b_{2} + ln$,
and multiply them to obtain
$a_{1}b_{1} = a_{2}b_{2} + a_{2}ln + b_{2}kn + kln^{2}$,
whence
\[ a_{1}b_{1} - a_{2}b_{2} = (a_{2}l + b_{2}k + kln)n ; \]
thus $n \divides (a_{1}b_{1} - a_{2}b_{2})$,
so $a_{1}b_{1} \equiv a_{2}b_{2}$.
\end{proof}

\begin{Def}
The {\em residue map modulo $n$} is the function
\[ \res_n : \bZ \to \bZ_n \quad \text{ given by } \quad \res_n(a) = \BAR{a} . \]
\end{Def}

\begin{Prop}
Let $n \in \bZ$, $n \ge 2$, and consider the residue map $\res_n : \bZ \to \bZ_n$.
Then
\begin{itemize}
\ITEM{a} $\res_n(0) = \BAR{0}$ and $\res_n(1) = \BAR{1}$;
\ITEM{b} $\res_n(a+b) = \res_n(a) + \res_n(b)$;
\ITEM{c} $\res_n(ab) = \res_n(a) \res_n(b)$.
\end{itemize}
\end{Prop}

\begin{proof}
This is immediate from the definitions of addition and multiplication in $\bZ_n$,
and the fact that the are well-defined.
\end{proof}

\newpage

\section{Properties of Addition}

\begin{Prop}
Addition on $\bZ_{n}$ is commutative, associative, admits an identity $\BAR{0}$,
and admits additive inverses.
\end{Prop}

\begin{proof}
Select $a,b \in \bZ$ so that $\BAR{a}$, $\BAR{b}$, and $\BAR{c}$
are arbitrary members of $\bZ_{n}$.

To see that $+$ is commutative, note that
\[ \BAR{a} + \BAR{b} = \BAR{a+b} = \BAR{b+a} = \BAR{b} + \BAR{a} . \]

To see that $+$ is associative, compute
\[ (\BAR{a}+\BAR{b}) + \BAR{c} = \BAR{a+b} + \BAR{c} = \BAR{(a+b)+c} = \BAR{a+(b+c)} = \BAR{a} + \BAR{b+c}
    = \BAR{a} + (\BAR{b} + \BAR{c}) . \]
%to see
%\begin{align}
%\BAR{a} + \BAR{b}
%        &= \BAR{a+b} \text{ by definition of } + \notag \\
%        &= \BAR{b+a} \text{ by commutativity in } \bZ \notag \\
%        &= \BAR{b} + \BAR{a} \notag
%\end{align}

%To see that $+$ is associative, note that
%\begin{align}
%(\BAR{a}+\BAR{b}) + \BAR{c}
%        &= \BAR{a+b} + \BAR{c} \notag \\
%        &= \BAR{(a+b)+c} \notag \\
%        &= \BAR{a+(b+c)} \notag \\
%        &= \BAR{a} + \BAR{b+c} \notag \\
%        &= \BAR{a} + (\BAR{b}+ \BAR{c}). \notag
%\end{align}

To see that $\BAR{0}$ is an additive identity, note that
$\BAR{0} + \BAR{a} = \BAR{0+a} = \BAR{a}$.

The additive inverse of $\BAR{a}$ is $\BAR{-a}$, since
$\BAR{a} + \BAR{-a} = \BAR{a-a} = \BAR{0}$.
\end{proof}

For any $k \in \bN$ and any $\BAR{a} \in \bZ_{n}$, define $k\BAR{a}$
to be $\BAR{a}$ added to itself $k$ times:
\[ k\BAR{a} = \sum_{i=1}^{k} \BAR{a} . \]

\begin{Prop}
Let $k \in \bN$ and $\BAR{a} \in \bZ_{n}$. Then $k\BAR{a} =
\BAR{ka}$.
\end{Prop}

\begin{proof}
%Since addition is associative, we can ignore parentheses.
%Then
$k\BAR{a} = \sum_{i=1}^{k} \BAR{a} = \BAR{\sum_{i=1}^{k} a} = \BAR{ka}$.
\end{proof}

In $\bZ_n$, we have $n \BAR{a} = \BAR{na} = \BAR{0}$.
So, some multiple of $\BAR{a}$ is zero; thus there is a smallest positive integer $k$
such that $k\BAR{a} = 0$.

\begin{Def}
Let $\BAR{a} \in \bZ_{n}$. Define the {\em additive order} of $\BAR{a}$ to be
smallest positive integer $k$ such that $k\BAR{a} = \BAR{0}$.  The additive
order of $\BAR{a}$ is denoted $\ord_+(\BAR{a})$.
\end{Def}

\begin{Prop}
Let $\BAR{a} \in \bZ_{n}$ and let $\ord_+(\BAR{a}) = k$.
Then \\
\BOLD{(a)} $j\BAR{a} = \BAR{0} \IFF k \divides j$; \\
\BOLD{(b)} $n\BAR{a} = \BAR{0}$; \\
\BOLD{(c)} $k \divides n$.
\end{Prop}

\begin{proof}
\text{ }

\BOLD{(a)} If $k \divides j$, then $j = lk$ for some $l \in \bZ$. In
this case, $j\BAR{a} = l\BAR{0} = \BAR{0}$.

Conversely, suppose that $j\BAR{a} = \BAR{0}$.
Write $j = qk + r$, where $0 \le r < k$.
Then $j\BAR{a} = qk\BAR{a} + r\BAR{a} = r\BAR{a}$ since $k\BAR{a} = 0$.
But $k$ is the smallest positive integer such that $k\BAR{a} = \BAR{0}$.
Thus $r = 0$, and $j = qk$.  Thus $k \divides j$.

\BOLD{(b)}
Note that $n\BAR{a} = \BAR{na} = \BAR{0}$.
Thus $n\BAR{a} = \BAR{0}$.

\BOLD{(c)}
By (b), $n\BAR{a} = \BAR{0}$.
Thus $k \divides n$ by part (a).
\end{proof}

\begin{Prop}
Let $\BAR{a} \in \bZ_{n}$ and let $d = \gcd(a,n)$. Then $\ord_+(\BAR{a}) = \frac{n}{d}$.
\end{Prop}

\begin{proof}
Let $k = \ord(\BAR{a})$.  Now $\frac{n}{d} \BAR{a} = \BAR{\frac{na}{d}} = n \BAR{\frac{a}{d}} = \BAR{0}$;
thus $k \divides \frac{n}{d}$.

On the other hand, $k\BAR{a} = \BAR{0}$, so $ka = nl$ for some $l \in \bZ$.
Dividing by $d$ gives $k\frac{a}{d} = \frac{n}{d} l$.
Thus $\frac{n}{d} \divides k \frac{a}{d}$, and since $\gcd(\frac{a}{d},\frac{n}{d}) = 1$,
we have $\frac{n}{d} \divides k$.

Thus $k \divides \frac{n}{d}$ and $\frac{n}{d} \divides k$, and since both are positive
they must be equal.
\end{proof}

\begin{Exm}
Let $n = 24$ and $a = 20$.
Now $\gcd(a,n) = 4$, so $\ord_+(a) = \frac{24}{4} = 6$.
Indeed, $6 \cdot 20 = 120$ is the smallest multiple of $20$
which is divisible by $24$.
\end{Exm}

\begin{Exm}
Let $p = 7$ and consider $\bZ_p$.  The order of every nonzero element is $7$.
\end{Exm}

\newpage

\section{Properties of Multiplication}

\begin{Prop}
Multiplication on $\bZ_{n}$ is commutative and associative, with
identity element $\BAR{1}$. Furthermore, multiplication distributes
over addition.
%\[ \BAR{a} \cdot (\BAR{b} + \BAR{c}) =
%        (\BAR{a}\cdot\BAR{b}) + (\BAR{a}\cdot\BAR{c}) \]
%for all $\BAR{a},\BAR{b},\BAR{c} \in \bZ$.
\end{Prop}

\begin{proof}
Select $a,b,c \in \bZ$ so that $\BAR{a}$, $\BAR{b}$, and $\BAR{c}$
are arbitrary members of $\bZ_{n}$.

To see that multiplication is commutative, compute
\[ \BAR{a}\cdot\BAR{b} = \BAR{ab} = \BAR{ba} = \BAR{b} \cdot \BAR{a} . \]

To see that multiplication is associative, compute
\[ (\BAR{a}\cdot\BAR{b})\cdot\BAR{c} = \BAR{ab}\cdot\BAR{c}
        = \BAR{abc} = \BAR{a}\cdot\BAR{bc}
        = \BAR{a}\cdot(\BAR{b}\cdot\BAR{c}) . \]

To see that $\BAR{1}$ is a multiplicative identity, compute
$\BAR{a}\cdot\BAR{1} = \BAR{a \cdot 1} = \BAR{a}
        = \BAR{1 \cdot a} = \BAR{1} \cdot \BAR{a}$.

To see the multiplication distributes over addition, compute
\[ \BAR{a}\cdot(\BAR{b}+\BAR{c}) = \BAR{a}\cdot\BAR{b+c}
        = \BAR{a(b+c)} = \BAR{ab+ac} = \BAR{ab}+\BAR{ac}
        = (\BAR{a}\cdot\BAR{b})+(\BAR{a}\cdot\BAR{c}) . \]
\end{proof}

\begin{Prop}
Let $\BAR{a} \in \bZ_{n}$. Then $\BAR{a}\cdot\BAR{0} =
\BAR{0}\cdot\BAR{a} = \BAR{0}$.
\end{Prop}

\begin{proof}
By definition of multiplication in $\bZ_{n}$, $\BAR{a} \cdot \BAR{0}
= \BAR{a \cdot 0}
        = \BAR{0} = \BAR{0 \cdot a} = \BAR{0} \cdot \BAR{a}$.
\end{proof}

\begin{Def}
Let $n \in \bZ$, $n \ge 2$, and let $\BAR{a} \in \bZ_n$.
We say that $\BAR{a}$ is {\em invertible} in $\bZ_n$ if there
exists an element $\BAR{b} \in \bZ_{n}$ such that
$\BAR{a}\cdot\BAR{b} = \BAR{1}$.
\end{Def}

\begin{Prop} \label{GCDInv}
Let $\BAR{a} \in \bZ_{n}$.
Then $\BAR{a}$ is invertible if and only if $\gcd(a,n) = 1$.
\end{Prop}

\begin{proof}
\text{ }

($\IMP$) Suppose that $\BAR{a}$ is invertible, and let $\BAR{b}$ be
its inverse. Then $\BAR{ab} = \BAR{1}$, so $ab \equiv 1 \MOD{n}$.
That is, $kn = ab-1$ for some $k \in \bZ$. Thus $ab + (-k)n = 1$.
By Proposition \ref{GCDOne}, $\gcd(a,n) = 1$.

($\IMB$) Suppose that $\gcd(a,n) = 1$. Then there exist $x,y \in
\bZ$ such that $xa + yn = 1$. Then $\BAR{x} \cdot \BAR{a} + \BAR{y}
\cdot \BAR{n} = \BAR{1}$. But $\BAR{n} = \BAR{0}$, so $\BAR{y} \cdot
\BAR{n} = \BAR{0}$. Thus $\BAR{x} \cdot \BAR{a} = \BAR{1}$, and
$\BAR{x}$ is the inverse of $\BAR{a}$, so $\BAR{a}$ is invertible.
\end{proof}

\begin{Exm}
Let $p \in \bN$ be a prime number. \\
Then every nonzero element of $\bZ_{p}$ is invertible, because each nonzero positive integer
less than $p$ is relatively prime to $p$.
\end{Exm}

\begin{Def}
Let $n \in \bZ$ with $n \ge 2$, and let $\BAR{a} \in \bZ_n$ be nonzero.
We say that $\BAR{a}$ is a {\em zero divisor} if
there exists $\BAR{b} \in \bZ_n$ which is nonzero
such that $\BAR{a} \BAR{b} = \BAR{0}$.
\end{Def}

\begin{Prop} \label{InvNotZD}
Let $n \in \bZ$ with $n \ge 2$, and let $\BAR{a} \in \bZ_n$.
If $\BAR{a}$ is invertible, then $\BAR{a}$ is not a zero divisor.
\end{Prop}

\begin{proof}
Suppose $a$ is invertible, and let $b \in \bZ$
such that $a b = 0$.
Multiply on the left by $a^{-1}$ to get
$a^{-1} a b = a^{-1} \cdot 0$,
whence $b = 0$.  This shows that $a$ is not a zero divisor,
because the only element in $\bZ_n$ which can be multiplied with $a$
to produce $0$ is $0$ itself.
\end{proof}

\begin{Exm}
Let $n = 6$; in $\bZ_{6}$, the invertible elements are $\BAR{1}$ and $\BAR{5}$.
The zero divisors are $\BAR{2}$, $\BAR{3}$, and $\BAR{4}$.
To see this, consider $\BAR{2} \cdot \BAR{3} = \BAR{6} = 0$,
and $\BAR{3} \cdot \BAR{4} = \BAR{12} = 0$.
\end{Exm}

\newpage

\begin{Prop}
Let $n \in \bZ$ with $n \ge 2$, and let $\BAR{a} \in \bZ_n$ be nonzero.
Then $\BAR{a}$ is a zero divisor if and only if $\gcd(a,n) \ge 2$.
\end{Prop}

\begin{proof}
Let $d = \gcd(a,n)$.

Suppose that $d = 1$.  Then $\BAR{a}$ is invertible by Proposition \ref{GCDInv}, so $\BAR{a}$ is not
a zero divisor by Proposition \ref{InvNotZD}.

Suppose that $d \ge 2$.  Using arithmetic in $\bZ$, the Euclidean algorithm
dictates that there exist $x,y \in \bZ$ such that $ax + ny = d$.
We also have $d \divides n$.
Then there exists $b \in \bZ$ such that $bd = n$, and since $d \ge 2$, we have $0 < b < n$.
Applying the residue map to $ax + ny = d$ gives $\BAR{a} \BAR{x} + \BAR{n} \BAR{y} = \BAR{d}$,
and since $\BAR{n} = \BAR{0}$, we have $\BAR{a} \BAR{x} = \BAR{d}$.
Multiply this equation by $\BAR{b}$ to get
\[ \BAR{a} \BAR{x} \BAR{b} = \BAR{db} = \BAR{n} = \BAR{0} . \]
Thus $\BAR{a}$ is a zero divisor.
\end{proof}

\begin{Def}
The {\em group of units} of $\bZ_n$ is
\[ \bZ_n^* = \{ \BAR{a} \in \bZ_n \mid \gcd(a,n) = 1 \} . \]
The {\em Euler phi function} is defined by $\phi(n) = \modu{\bZ_n^*}$.
\end{Def}

Thus $\BAR{a} \in \bZ_n^*$ if and only if $\BAR{a}$ is invertible in $\bZ_n$.
The next proposition says that $\bZ_n^*$ is closed under multiplication.

\begin{Prop}
Let $n \in \bZ$, $n \ge 2$, and let $\BAR{a},\BAR{b} \in \bZ$ be invertible.
Then $\BAR{a} \BAR{b}$ is invertible.
\end{Prop}

\begin{proof}
Clearly, $(\BAR{a} \BAR{b}) = \BAR{b}^{-1} \BAR{a}^{-1}$, since
$(\BAR{a} \BAR{b})(\BAR{b}^{-1} \BAR{a}^{-1}) = \BAR{a} ( \BAR{b} \BAR{b}^{-1} ) \BAR{a}^{-1}
 %   = \BAR{a} \cdot \BAR{1} \cdot \BAR{a}^{-1}
    = \BAR{a} \BAR{a}^{-1} = \BAR{1}$.
\end{proof}

For example,
\begin{itemize}
\item $\bZ_p^* = \{ 1, \dots, p-1 \}$, if $p$ is prime;
\item $\bZ_6^* = \{ 1,5 \}$;
\item $\bZ_{12}^* = \{ 1,5,7,11 \}$;
\item $\bZ_{15}^* = \{ 1,2,4,6,7,8,11,13,14 \}$.
\end{itemize}

\begin{comment}
\begin{Def}
The {\em Euler phi function} is defined by
\[ \phi : \bN \to \bN \quad \text{ given by } \quad
\phi(n) =
\begin{cases}
    0   \quad &\text{ if $n = 0$}; \\
    1   \quad &\text{ if $n = 0$};\\
    \modu{\bZ_n^*} \quad &\text{ if $n \ge 2$}.
\end{cases}
\]
\end{Def}

Thus, for $n \ge 2$, $\phi(n)$ is the number of positive integers less than $n$ which are relatively
prime to $n$.  For example,
\[ \phi(2) = 1, \; \phi(3) = 2, \; \phi(4) = 2 , \; \phi(5) = 4 , \; \phi(6) = 2 , \; \phi(7) = 6,
    \; \phi(8) = 4 , \; \phi(9) = 6 . \]
In general, if $p$ is prime, then $\phi(p) = p-1$.
\end{comment}

\begin{Def}
Let $n \in \bZ$, $n \ge 2$, and let $\BAR{a} \in \bZ_n^*$.
The {\em multiplicative order} of $\BAR{a}$, denoted $\ord_*(\BAR{a})$ is the smallest positive integer $k$
such that $\BAR{a}^k = \BAR{1}$.
\end{Def}

\begin{Exm}
Find $\ord_*(\BAR{7})$ in $\bZ_{15}^*$.
\end{Exm}

\begin{proof}[Solution]
We have
\begin{align*}
    \BAR{7}^2 &= \BAR{49} = \BAR{4} ; \\
    \BAR{7}^3 &= \BAR{4} \cdot \BAR{7} = \BAR{28} = \BAR{13} ; \\
    \BAR{7}^4 &= \BAR{13} \cdot \BAR{7} = \BAR{91} = \BAR{1} .
\end{align*}
Thus $\ord_*(\BAR{7}) = 4$.
\end{proof}

\begin{comment}
\newpage

\begin{Prop} \BOLD{(Euler's Theorem)} \\
Let $a,n \in \bZ$ with $n \ge 2$ and $\gcd(a,n) = 1$.
Then
\[ a^{\phi(n)} \equiv 1 \MOD{n} . \]
\end{Prop}

\begin{proof}
This is equivalent to showing that $\BAR{a}^{\phi(n)} = \BAR{1}$.
Since $\gcd(a,n) = 1$, $\BAR{a}$ is invertible and $\BAR{a} \in \bZ_n^*$.

Consider the function $f :\bN \to \bZ_n^*$ given by $f(i) = \BAR{a}^i$.
Since $\bN$ is infinite and $\bZ_n^*$ is finite, this function cannot be injective;
let $j$ be the smallest positive integer such that $f(j) = f(k)$ for some $k > j$.
Then $\BAR{a}^j = \BAR{a}^k$, and since $\BAR{a}$ is invertible, so is $\BAR{a}^{j}$,
with inverse $\BAR{a}^{-j}$.
Thus $\BAR{a}^{k-j} = \BAR{1}$.

Consider the function $\alpha : \bZ_n^* \to \bZ_n^*$ given by $\alpha(\BAR{x}) = \BAR{a}\BAR{x}$.
This function has an inverse $\BAR{x} \mapsto \BAR{a}^{-1} \BAR{x}$,
so it is bijective.
\end{comment}

\newpage

\section{Casting Out $n$'s}

The process of {\em casting out $n$'s} involves subtracting $n$ from a number
until one arrives at a number less than $n$.  Clearly, this number is the remainder
upon division by $n$, so it is related to modular arithmetic.

The method of casting out $n$'s, together with decimal notation,
led Arabs of 1500 years ago to discover certain divisibility criteria.
We demonstrate this in modern notation.

Fix $n \in \bZ$ with $n \ge 0$.  For $a \in \bZ$, let $\BAR{a}$ denote the remainder when
$a$ is divide by $n$.  The last proposition states that $\BAR{a+b} \equiv \BAR{a} + \BAR{b}$
and $\BAR{ab} \equiv \BAR{a} \BAR{b}$, modulo $n$.

If $d_0,d_1, \dots, d_r$ are the digits of $a \in \bN$ (where $0 \le d_i \le 9$), then
\[ a = \sum_{i=0}^r d_i \cdot 10^i . \]
The idea of casting out $n$'s revolves around the fact that
\[ a \equiv \sum_{i=0}^r \BAR{d_i} \cdot \BAR{10}^i \MOD{n} . \]

\begin{Prop}
\BOLD{(Casting Out $3$'s and $9$'s)} \\
Let $n = 3$ or $n = 9$.  Let $a,s \in \bZ$ be given by
\[ a = \sum_{i=0}^{k} d_i \cdot 10^i \quad \text{ and } \quad s = \sum_{i=0}^k d_i . \]
Then $a$ is divisible by $n$ if and only if $s$ is divisible by $n$.
\end{Prop}

\begin{proof}
In $\bZ_3$ or $\bZ_9$, we have $\BAR{10} = \BAR{1}$.
Thus
\[ \BAR{a} = \BAR{\sum_{i=0}^k d_i \cdot 10^i}
    = \sum_{i=0}^k \BAR{d_i} \cdot \BAR{10}^i \\
    = \sum_{i=0}^k \BAR{d_i}
    = \BAR{s}. \]
So $a$ and $s$ have the same remainder upon division by $n$,
and in particular $a$ is divisible by $n$ if and only if $s$ is
divisible by $n$.
\end{proof}

\begin{Prop}
\BOLD{(Casting Out $11$'s)} \\
Let $n = 11$.  Let $a,s \in \bZ$ be given by
\[ a = \sum_{i=0}^{k} d_i \cdot 10^i \quad \text{ and } \quad s = \sum_{i=0}^k (-1)^i d_i . \]
Then $a$ is divisible by $n$ if and only if $s$ is divisible by $n$.
\end{Prop}

\begin{proof}
In $\bZ_{11}$, we have $10 \equiv -1 \MOD{n}$.
Thus
\[ \BAR{a} = \BAR{\sum_{i=0}^k d_i \cdot 10^i}
    = \sum_{i=0}^k \BAR{d_i} \cdot \BAR{10}^i
    = \sum_{i=0}^k \BAR{d_i} (\BAR{-1})^i \\
    = \BAR{s}. \]
Thus $a$ is divisible by $n$ if and only if $s$ is divisible by $n$.
\end{proof}


\newpage

\section{Algebraic Equations in $\bZ_{n}$}

We now turn our attention to the question of when an equation,
such as $\BAR{14}x = \BAR{1}$ or $x^{2}+\BAR{1} = \BAR{0}$, has a solution in $\bZ_{n}$,
and how many solutions it has.
For example, $\BAR{14}x = \BAR{1}$ has a solution if and only if $\BAR{14}$ is invertible
in $\bZ_{n}$, and this is the case if and only if $n$ and $14$
are relatively prime.  In fact, we have an explicit technique for
finding the inverse $\BAR{14}$.  This technique makes repeated use of
the division algorithm.

Suppose $n = 33$.  Then $14$ and $33$ are relatively prime,
so there exist integers $x$ and $y$ such that $14x  + 33y = 1$.
To find them, we divide:
\begin{itemize}
\item $33 = 14 \cdot 2 + 5$;
\item $14 = 5 \cdot 2 + 4$
\item $5 = 4 \cdot 1 + 1$;
\item $2 = 1 \cdot 2 + 0$.
\end{itemize}

The second to last remainder is $1$, so $\gcd(14,33) = 1$.
Now work backwards to find $x$ and $y$:
\begin{itemize}
\item $1 = 5 - 4$;
\item $1 = 5 - (14-5 \cdot 2) = 5 \cdot 3 - 14 \cdot 1$;
\item $1 = (33 - 14 \cdot 2) \cdot 3 - 14 \cdot 1 = 33 \cdot 3 - 14 \cdot 7$.
\end{itemize}

Thus the inverse of $\BAR{14}$ in $\bZ_{33}$ is $\BAR{-7} = \BAR{26}$.

The equation $x^{2}+\BAR{1} = \BAR{0}$
is more interesting.  To understand it, note
that negative $\BAR{1}$ exists in $\bZ_{n}$ as $\BAR{n-1}$.
So a solution to the equation $x^{2} + \BAR{1} = \BAR{0}$ would be a square root of
negative $\BAR{1}$ in $\bZ_{n}$.
%It can be shown that when $n$ is prime,
%then this has a solution
%if and only if $4 \divides (n-1)$.
For example, in $\bZ_{5}$, we have $\BAR{2}^{2} = \BAR{4} = -\BAR{1}$.

It is also possible that a quadratic equation, such as $x^{2} - \BAR{1} = \BAR{0}$,
can have more than two solutions in $\bZ_{n}$.
Note that $x^{2} - \BAR{1} = (x+\BAR{1})(x-\BAR{1})$, even in $\bZ_{n}$.
Suppose that $n = 15$.
Then $x = \BAR{1}$ and $x = -\BAR{1} = \BAR{14}$ are solutions, but so is $\BAR{4}$,
since $(\BAR{4}+\BAR{1})(\BAR{4}-\BAR{1}) = \BAR{5} \cdot \BAR{3} = \BAR{0}$ in $\bZ_{15}$.

However, suppose that $n = p$ is a prime number.
Then in $\bZ_{p}$, a quadratic equation can have at most $2$ roots.
This is because $\bZ_{p}$ has no zero divisors.
If the quadratic has a root, it factors; then if the product of
the factors is zero, one of them must be zero.

For example, let us find the roots of $x^{2} + \BAR{8}x + \BAR{1} = \BAR{0}$ in $\bZ_{11}$.
Now $8 \equiv -3 \MOD{11}$ and $1 \equiv -10 \MOD{11}$,
so our equation becomes $x^{2} - \BAR{3}x - \BAR{10} = \BAR{0}$.
This factors as $(x-\BAR{5})(x+\BAR{2}) = 0$.  Since $11$ is prime,
the only roots are $\BAR{8}$ and $-\BAR{2} = \BAR{8}$.

\newpage

\section{Exercises}

\begin{Exc}
Let $a,b,c \in \bN$ be positive.  Show that
\begin{itemize}
\ITEM{a} $a \divides a$; \ITEM{b} $a \divides b$ and $b \divides
a$ implies $a = b$; \ITEM{c} $a \divides b$ and $b \divides c$
implies $a \divides c$.
\end{itemize}
\end{Exc}

\begin{Exc}
Let $m,n,d \in \bZ$ with $d = \gcd(m,n)$.
Show that %$\lcm(m,n)$ exists, and in fact,
\[ \lcm(m,n) = \frac{mn}{d} . \]
\end{Exc}

\begin{Exc}
Construct a proof of the Euclidean Algorithm using
%, using Proposition \ref{GCDObvious},
Proposition \ref{GCDRecursive} and induction.
\end{Exc}

\begin{Exc}
Let $n \in \bZ$ with $n \ge 2$.  Show that if $n$ is not a prime number,
then $\bZ_{n}$ contains zero divisors.
\end{Exc}

\begin{Exc}
Let $n \in \bZ$ with $n \ge 2$, and let $\BAR{a} \in \bZ_{n}$ be a nonzero element.
Show that $\BAR{a}$ is invertible if and only if $\BAR{a}$ is not a zero divisor.
\end{Exc}

\begin{Exc}
Find the additive order of $\BAR{6}$, $\BAR{11}$, $\BAR{18}$, and $\BAR{28}$ in $\bZ_{36}$.
\end{Exc}

\begin{Exc}
Find $\bZ_{48}^*$.
\end{Exc}

\begin{Exc}
Find $\phi(100)$.
\end{Exc}

\begin{Exc}
Find the multiplicative order of $\BAR{10}$ in $\bZ_{21}^*$.
\end{Exc}

\begin{Exc}
Find the inverse of $\BAR{15}$ in $\bZ_{49}$.
\end{Exc}

\begin{Exc}
Solve the equation $\BAR{17}x = \BAR{23}$ in $\bZ_{71}$.
\end{Exc}

\begin{Exc}
Solve the equation $x^2 - \BAR{5}x - \BAR{2} = \BAR{0}$ in $\bZ_{11}$.
\end{Exc}

\begin{Exc}
Solve the equation $x^2 - \BAR{5}x + \BAR{4} = 0$ in $\bZ_{6}$.
\end{Exc}

\begin{Exc}
Find all square roots of $-\BAR{1}$ in $\bZ_{101}$.
\end{Exc}

\end{document}
